%%%%%%%%%%%%%%%%%%%%%%%%%%%%%%%%%%%%%%%%%%%%%%%%%%%%%%%%%%%%%%%%%%%%%%
\clearpage

\section{Conclusions}
\label{Conclusions}

In this document we have explained how MMT works. The MMT-DPI library is a commercial library to classify and decode network packets et flows. It has a limited release that is distributed as freeware to be used with the open source version of the MMT-Security library. The goal of Montimage is to show how the MMT-DPI library can be used in combination with security properties to detect  functional and security related behaviours and alert incidents. The technique presented in this tutorial can be applied for the analysis of any type of events producing structured information. Theses events can be telecommunication packets (as shown in this document), log entries, application messages, traces etc. The analysis itself can done at runtime or offline using a precaptured trace.    

% \subsection{Development Roadmap}
% The MMT\_Security library is a first version that will be continuously improved and expanded. Note that many of the features listed below are already implemented in the commercial version of MMT\_Security.

% Improvements planned for the open source MMT\_Security version are:

% \begin{itemize}
% \item General improvements:
%     \begin{itemize}
%     \item Change to object language (e.g., C++) and create hierarchy for tree nodes;
%     \item Comment code;
%     \end{itemize}

% \item New features planned are:
%     \begin{itemize}
%     \item Concerning the probes:

%         \begin{itemize}
%         \item Remote control of probes (API for management);
%         \item Graphic representation of results and alarms. Interact with a centralised operator application (e.g., MMT\_Operator).
%         \end{itemize}

%     \item Concerning distributed analysis:

%         \begin{itemize}
%         \item Collaboration between distributed probes
%         \item Allow identification of neighbours (e.g., specific, sets, types) in rules;
%         \item Synchronize observation points (e.g., NTP protocol);
%         \item Combine distributed and centralised security analysis.
%         \end{itemize}

%     \item Concerning the security properties:

%         \begin{itemize}
%         \item Specification of repeated occurrence of events;
%         \item On-the-fly addition or elimination of security properties;
%         \item Complete missing data types;
%         \item Build user friendly editor for specifying security properties;
%         \item Property correctness verifier;
%         \item Limit the number of verdicts when many related ones are possible;
%         \item Report inconclusive verdicts. At the end or beginning of a file, could have non-conclusive properties (i.e., packets needed for making verdicts are missing at the beginning or at the end);
%         \item Allow use of session information;
%         \item Improve specifications of time delays;
%         \item Implement delays with respect to the number of packets; 
%         \item Imbricated protocols (e.g. IPV6.ZZZ.YYY);
%         \item Improve the expressiveness of the properties; Improve boolean expression (eliminate parentheses, generalise boolean expressions...);
%         \item Translate to and from other formalisms: Bro, Snort, TTCN3, SDL...
%         \end{itemize}

% \item Performance improvements:
%         \begin{itemize}
%         \item Don't create instances if not needed;
%         \item First check available events (next events to check) then see if rule is still respected. This will allow cross-references and avoiding repeated verification of same events;
%         \item Use hash tables, multi-threads, multi-core...;
%         \item Improve filtering.
%         \end{itemize}
% \end{itemize}
% \end{itemize}

For further information, please send an email to \url{contact@montimage.com}.
%%%%%%%%%%%%%%%%%%%%%%%%%%%%%%%%%%%%%%%%%%%%%%%%%%%%%%%%%%%%%%%%%%%%%%
